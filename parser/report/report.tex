
%% bare_conf.tex
%% V1.3
%% 2007/01/11
%% by Michael Shell
%% See:
%% http://www.michaelshell.org/
%% for current contact information.
%%
%% This is a skeleton file demonstrating the use of IEEEtran.cls
%% (requires IEEEtran.cls version 1.7 or later) with an IEEE conference paper.
%%
%% Support sites:
%% http://www.michaelshell.org/tex/ieeetran/
%% http://www.ctan.org/tex-archive/macros/latex/contrib/IEEEtran/
%% and
%% http://www.ieee.org/

%%*************************************************************************
%% Legal Notice:
%% This code is offered as-is without any warranty either expressed or
%% implied; without even the implied warranty of MERCHANTABILITY or
%% FITNESS FOR A PARTICULAR PURPOSE! 
%% User assumes all risk.
%% In no event shall IEEE or any contributor to this code be liable for
%% any damages or losses, including, but not limited to, incidental,
%% consequential, or any other damages, resulting from the use or misuse
%% of any information contained here.
%%
%% All comments are the opinions of their respective authors and are not
%% necessarily endorsed by the IEEE.
%%
%% This work is distributed under the LaTeX Project Public License (LPPL)
%% ( http://www.latex-project.org/ ) version 1.3, and may be freely used,
%% distributed and modified. A copy of the LPPL, version 1.3, is included
%% in the base LaTeX documentation of all distributions of LaTeX released
%% 2003/12/01 or later.
%% Retain all contribution notices and credits.
%% ** Modified files should be clearly indicated as such, including  **
%% ** renaming them and changing author support contact information. **
%%
%% File list of work: IEEEtran.cls, IEEEtran_HOWTO.pdf, bare_adv.tex,
%%                    bare_conf.tex, bare_jrnl.tex, bare_jrnl_compsoc.tex
%%*************************************************************************

% *** Authors should verify (and, if needed, correct) their LaTeX system  ***
% *** with the testflow diagnostic prior to trusting their LaTeX platform ***
% *** with production work. IEEE's font choices can trigger bugs that do  ***
% *** not appear when using other class files.                            ***
% The testflow support page is at:
% http://www.michaelshell.org/tex/testflow/



% Note that the a4paper option is mainly intended so that authors in
% countries using A4 can easily print to A4 and see how their papers will
% look in print - the typesetting of the document will not typically be
% affected with changes in paper size (but the bottom and side margins will).
% Use the testflow package mentioned above to verify correct handling of
% both paper sizes by the user's LaTeX system.
%
% Also note that the "draftcls" or "draftclsnofoot", not "draft", option
% should be used if it is desired that the figures are to be displayed in
% draft mode.
%
\documentclass[conference]{IEEEtran}
% Add the compsoc option for Computer Society conferences.
%
% If IEEEtran.cls has not been installed into the LaTeX system files,
% manually specify the path to it like:
% \documentclass[conference]{../sty/IEEEtran}





% Some very useful LaTeX packages include:
% (uncomment the ones you want to load)


% *** MISC UTILITY PACKAGES ***
%
%\usepackage{ifpdf}
% Heiko Oberdiek's ifpdf.sty is very useful if you need conditional
% compilation based on whether the output is pdf or dvi.
% usage:
% \ifpdf
%   % pdf code
% \else
%   % dvi code
% \fi
% The latest version of ifpdf.sty can be obtained from:
% http://www.ctan.org/tex-archive/macros/latex/contrib/oberdiek/
% Also, note that IEEEtran.cls V1.7 and later provides a builtin
% \ifCLASSINFOpdf conditional that works the same way.
% When switching from latex to pdflatex and vice-versa, the compiler may
% have to be run twice to clear warning/error messages.






% *** CITATION PACKAGES ***
%
%\usepackage{cite}
% cite.sty was written by Donald Arseneau
% V1.6 and later of IEEEtran pre-defines the format of the cite.sty package
% \cite{} output to follow that of IEEE. Loading the cite package will
% result in citation numbers being automatically sorted and properly
% "compressed/ranged". e.g., [1], [9], [2], [7], [5], [6] without using
% cite.sty will become [1], [2], [5]--[7], [9] using cite.sty. cite.sty's
% \cite will automatically add leading space, if needed. Use cite.sty's
% noadjust option (cite.sty V3.8 and later) if you want to turn this off.
% cite.sty is already installed on most LaTeX systems. Be sure and use
% version 4.0 (2003-05-27) and later if using hyperref.sty. cite.sty does
% not currently provide for hyperlinked citations.
% The latest version can be obtained at:
% http://www.ctan.org/tex-archive/macros/latex/contrib/cite/
% The documentation is contained in the cite.sty file itself.






% *** GRAPHICS RELATED PACKAGES ***

\ifCLASSINFOpdf
  \usepackage[pdftex]{graphicx}
%   declare the path(s) where your graphic files are
   \graphicspath{{./figures/}}
%   and their extensions so you won't have to specify these with
%   every instance of \includegraphics
   \DeclareGraphicsExtensions{.pdf,.jpeg,.png}
\else
  % or other class option (dvipsone, dvipdf, if not using dvips). graphicx
  % will default to the driver specified in the system graphics.cfg if no
  % driver is specified.
   \usepackage[dvips]{graphicx}
  % declare the path(s) where your graphic files are
   \graphicspath{{../eps/}}
  % and their extensions so you won't have to specify these with
  % every instance of \includegraphics
   \DeclareGraphicsExtensions{.eps}
\fi
% graphicx was written by David Carlisle and Sebastian Rahtz. It is
% required if you want graphics, photos, etc. graphicx.sty is already
% installed on most LaTeX systems. The latest version and documentation can
% be obtained at: 
% http://www.ctan.org/tex-archive/macros/latex/required/graphics/
% Another good source of documentation is "Using Imported Graphics in
% LaTeX2e" by Keith Reckdahl which can be found as epslatex.ps or
% epslatex.pdf at: http://www.ctan.org/tex-archive/info/
%
% latex, and pdflatex in dvi mode, support graphics in encapsulated
% postscript (.eps) format. pdflatex in pdf mode supports graphics
% in .pdf, .jpeg, .png and .mps (metapost) formats. Users should ensure
% that all non-photo figures use a vector format (.eps, .pdf, .mps) and
% not a bitmapped formats (.jpeg, .png). IEEE frowns on bitmapped formats
% which can result in "jaggedy"/blurry rendering of lines and letters as
% well as large increases in file sizes.
%
% You can find documentation about the pdfTeX application at:
% http://www.tug.org/applications/pdftex





% *** MATH PACKAGES ***
%
%\usepackage[cmex10]{amsmath}
% A popular package from the American Mathematical Society that provides
% many useful and powerful commands for dealing with mathematics. If using
% it, be sure to load this package with the cmex10 option to ensure that
% only type 1 fonts will utilized at all point sizes. Without this option,
% it is possible that some math symbols, particularly those within
% footnotes, will be rendered in bitmap form which will result in a
% document that can not be IEEE Xplore compliant!
%
% Also, note that the amsmath package sets \interdisplaylinepenalty to 10000
% thus preventing page breaks from occurring within multiline equations. Use:
%\interdisplaylinepenalty=2500
% after loading amsmath to restore such page breaks as IEEEtran.cls normally
% does. amsmath.sty is already installed on most LaTeX systems. The latest
% version and documentation can be obtained at:
% http://www.ctan.org/tex-archive/macros/latex/required/amslatex/math/





% *** SPECIALIZED LIST PACKAGES ***
%
\usepackage{algorithm}
\usepackage{algorithmic}
% algorithmic.sty was written by Peter Williams and Rogerio Brito.
% This package provides an algorithmic environment fo describing algorithms.
% You can use the algorithmic environment in-text or within a figure
% environment to provide for a floating algorithm. Do NOT use the algorithm
% floating environment provided by algorithm.sty (by the same authors) or
% algorithm2e.sty (by Christophe Fiorio) as IEEE does not use dedicated
% algorithm float types and packages that provide these will not provide
% correct IEEE style captions. The latest version and documentation of
% algorithmic.sty can be obtained at:
% http://www.ctan.org/tex-archive/macros/latex/contrib/algorithms/
% There is also a support site at:
% http://algorithms.berlios.de/index.html
% Also of interest may be the (relatively newer and more customizable)
% algorithmicx.sty package by Szasz Janos:
% http://www.ctan.org/tex-archive/macros/latex/contrib/algorithmicx/




% *** ALIGNMENT PACKAGES ***
%
%\usepackage{array}
% Frank Mittelbach's and David Carlisle's array.sty patches and improves
% the standard LaTeX2e array and tabular environments to provide better
% appearance and additional user controls. As the default LaTeX2e table
% generation code is lacking to the point of almost being broken with
% respect to the quality of the end results, all users are strongly
% advised to use an enhanced (at the very least that provided by array.sty)
% set of table tools. array.sty is already installed on most systems. The
% latest version and documentation can be obtained at:
% http://www.ctan.org/tex-archive/macros/latex/required/tools/


%\usepackage{mdwmath}
%\usepackage{mdwtab}
% Also highly recommended is Mark Wooding's extremely powerful MDW tools,
% especially mdwmath.sty and mdwtab.sty which are used to format equations
% and tables, respectively. The MDWtools set is already installed on most
% LaTeX systems. The lastest version and documentation is available at:
% http://www.ctan.org/tex-archive/macros/latex/contrib/mdwtools/


% IEEEtran contains the IEEEeqnarray family of commands that can be used to
% generate multiline equations as well as matrices, tables, etc., of high
% quality.


%\usepackage{eqparbox}
% Also of notable interest is Scott Pakin's eqparbox package for creating
% (automatically sized) equal width boxes - aka "natural width parboxes".
% Available at:
% http://www.ctan.org/tex-archive/macros/latex/contrib/eqparbox/





% *** SUBFIGURE PACKAGES ***
%\usepackage[tight,footnotesize]{subfigure}
% subfigure.sty was written by Steven Douglas Cochran. This package makes it
% easy to put subfigures in your figures. e.g., "Figure 1a and 1b". For IEEE
% work, it is a good idea to load it with the tight package option to reduce
% the amount of white space around the subfigures. subfigure.sty is already
% installed on most LaTeX systems. The latest version and documentation can
% be obtained at:
% http://www.ctan.org/tex-archive/obsolete/macros/latex/contrib/subfigure/
% subfigure.sty has been superceeded by subfig.sty.



%\usepackage[caption=false]{caption}
%\usepackage[font=footnotesize]{subfig}
% subfig.sty, also written by Steven Douglas Cochran, is the modern
% replacement for subfigure.sty. However, subfig.sty requires and
% automatically loads Axel Sommerfeldt's caption.sty which will override
% IEEEtran.cls handling of captions and this will result in nonIEEE style
% figure/table captions. To prevent this problem, be sure and preload
% caption.sty with its "caption=false" package option. This is will preserve
% IEEEtran.cls handing of captions. Version 1.3 (2005/06/28) and later 
% (recommended due to many improvements over 1.2) of subfig.sty supports
% the caption=false option directly:
%\usepackage[caption=false,font=footnotesize]{subfig}
%
% The latest version and documentation can be obtained at:
% http://www.ctan.org/tex-archive/macros/latex/contrib/subfig/
% The latest version and documentation of caption.sty can be obtained at:
% http://www.ctan.org/tex-archive/macros/latex/contrib/caption/




% *** FLOAT PACKAGES ***
%
%\usepackage{fixltx2e}
% fixltx2e, the successor to the earlier fix2col.sty, was written by
% Frank Mittelbach and David Carlisle. This package corrects a few problems
% in the LaTeX2e kernel, the most notable of which is that in current
% LaTeX2e releases, the ordering of single and double column floats is not
% guaranteed to be preserved. Thus, an unpatched LaTeX2e can allow a
% single column figure to be placed prior to an earlier double column
% figure. The latest version and documentation can be found at:
% http://www.ctan.org/tex-archive/macros/latex/base/



%\usepackage{stfloats}
% stfloats.sty was written by Sigitas Tolusis. This package gives LaTeX2e
% the ability to do double column floats at the bottom of the page as well
% as the top. (e.g., "\begin{figure*}[!b]" is not normally possible in
% LaTeX2e). It also provides a command:
%\fnbelowfloat
% to enable the placement of footnotes below bottom floats (the standard
% LaTeX2e kernel puts them above bottom floats). This is an invasive package
% which rewrites many portions of the LaTeX2e float routines. It may not work
% with other packages that modify the LaTeX2e float routines. The latest
% version and documentation can be obtained at:
% http://www.ctan.org/tex-archive/macros/latex/contrib/sttools/
% Documentation is contained in the stfloats.sty comments as well as in the
% presfull.pdf file. Do not use the stfloats baselinefloat ability as IEEE
% does not allow \baselineskip to stretch. Authors submitting work to the
% IEEE should note that IEEE rarely uses double column equations and
% that authors should try to avoid such use. Do not be tempted to use the
% cuted.sty or midfloat.sty packages (also by Sigitas Tolusis) as IEEE does
% not format its papers in such ways.





% *** PDF, URL AND HYPERLINK PACKAGES ***
%
%\usepackage{url}
% url.sty was written by Donald Arseneau. It provides better support for
% handling and breaking URLs. url.sty is already installed on most LaTeX
% systems. The latest version can be obtained at:
% http://www.ctan.org/tex-archive/macros/latex/contrib/misc/
% Read the url.sty source comments for usage information. Basically,
% \url{my_url_here}.





% *** Do not adjust lengths that control margins, column widths, etc. ***
% *** Do not use packages that alter fonts (such as pslatex).         ***
% There should be no need to do such things with IEEEtran.cls V1.6 and later.
% (Unless specifically asked to do so by the journal or conference you plan
% to submit to, of course. )


% correct bad hyphenation here
\hyphenation{op-tical net-works semi-conduc-tor}


\begin{document}
%
% paper title
% can use linebreaks \\ within to get better formatting as desired
\title{Evolving a Better Branch Predictor}


% author names and affiliations
% use a multiple column layout for up to three different
% affiliations
\author{\IEEEauthorblockN{Dan Zhang}
\IEEEauthorblockA{Department of Electrical and\\Computer Engineering\\
University of Texas at Austin\\
Email: dan.zhang@utexas.edu}
\and
\IEEEauthorblockN{Ben (Ching-Pei) Lin}
\IEEEauthorblockA{Department of Electrical and\\Computer Engineering\\
University of Texas at Austin\\
Email: bencplin@utexas.edu}
}
%\and
%\IEEEauthorblockN{James Kirk\\ and Montgomery Scott}
%\IEEEauthorblockA{Starfleet Academy\\
%San Francisco, California 96678-2391\\
%Telephone: (800) 555--1212\\
%Fax: (888) 555--1212}}

% conference papers do not typically use \thanks and this command
% is locked out in conference mode. If really needed, such as for
% the acknowledgment of grants, issue a \IEEEoverridecommandlockouts
% after \documentclass

% for over three affiliations, or if they all won't fit within the width
% of the page, use this alternative format:
% 
%\author{\IEEEauthorblockN{Michael Shell\IEEEauthorrefmark{1},
%Homer Simpson\IEEEauthorrefmark{2},
%James Kirk\IEEEauthorrefmark{3}, 
%Montgomery Scott\IEEEauthorrefmark{3} and
%Eldon Tyrell\IEEEauthorrefmark{4}}
%\IEEEauthorblockA{\IEEEauthorrefmark{1}School of Electrical and Computer Engineering\\
%Georgia Institute of Technology,
%Atlanta, Georgia 30332--0250\\ Email: see http://www.michaelshell.org/contact.html}
%\IEEEauthorblockA{\IEEEauthorrefmark{2}Twentieth Century Fox, Springfield, USA\\
%Email: homer@thesimpsons.com}
%\IEEEauthorblockA{\IEEEauthorrefmark{3}Starfleet Academy, San Francisco, California 96678-2391\\
%Telephone: (800) 555--1212, Fax: (888) 555--1212}
%\IEEEauthorblockA{\IEEEauthorrefmark{4}Tyrell Inc., 123 Replicant Street, Los Angeles, California 90210--4321}}




% use for special paper notices
%\IEEEspecialpapernotice{(Invited Paper)}




% make the title area
\maketitle


\begin{abstract}
%\boldmath

\end{abstract}
% IEEEtran.cls defaults to using nonbold math in the Abstract.
% This preserves the distinction between vectors and scalars. However,
% if the conference you are submitting to favors bold math in the abstract,
% then you can use LaTeX's standard command \boldmath at the very start
% of the abstract to achieve this. Many IEEE journals/conferences frown on
% math in the abstract anyway.

% no keywords




% For peer review papers, you can put extra information on the cover
% page as needed:
% \ifCLASSOPTIONpeerreview
% \begin{center} \bfseries EDICS Category: 3-BBND \end{center}
% \fi
%
% For peerreview papers, this IEEEtran command inserts a page break and
% creates the second title. It will be ignored for other modes.
\IEEEpeerreviewmaketitle



\section{Introduction}
% no \IEEEPARstart
\subsection{Background}
Modern microprocessors aggresively exploit instruction-level parallelism (ILP) through deep pipelines, superscalar issue, and out-of-order (OoO) execution. Such processors require a constant stream of instructions to execute in order to achieve maximum performance. However, branch hazards can disrupt the instruction stream because the correct order of instructions cannot be determined until the given branch is resolved.

Consequently, modern microprocessors rely heavily on \emph{speculative execution}. They aim to predict the direction and target of a branch and speculatively fetch the next instructions before the true order of instructions is known. Once the branch is resolved, if the wrong instructions were executed, then the processor must roll back to the previous state before speculative execution, and re-execute with the correct instructions. 

The process of predicting the direction of a branch is called \emph{branch prediction}. Many branch prediction strategies exist, with the most basic strategies being \emph{static}, meaning that the branch predictions are made at compile-time and remain static during execution. For example, one can predict that all backward branches will be taken, while all forward branches will not be taken \cite{Smith98}, under the rationale that loop branches jump backwards and are usually taken. 

More advanced branch prediction strategies are \emph{dynamic} and use run-time history to predict future branches. The pioneering dynamic branch predictor was Yeh and Patt's \emph{Two-Level Adaptive Predictor} \cite{Patt91}, which uses both a first-level branch history information, and a second-level pattern history, to make a prediction. The two-level adaptive predictor was very successful and was able to achieve an average of 97\% accuracy on benchmarks at the time \cite{Patt91}. Many other variations of the two-level adaptive predictor exist, and are differentiated by whether the branch history is global, per-address, or per-set, and whether the pattern history is global, per-address, or per-set \cite{Patt93}. 

One branch predictor may work best for a particular execution pattern, while another predictor may be better for another. \emph{Hybrid} branch predictors exploit this property by combining several predictors into a single predictor through the use of  a \emph{meta-predictor}, which remembers what predictor worked best for a given branch, so that the most suitable predictor is always used \cite{McFarling93}. 

Researchers are still coming up with newer and better designs for branch predictors. Some of the more recent designs, such as Jim\'{e}nez and Lin's \emph{perceptron predictor} \cite{Lin01}, have been based on machine learning techniques. The Championship Branch Prediction competition, which has been held in 2004, 2006, and 2011 \cite{championship}, is a competition in which contestants aim to design the most accurate branch predictor. The competition has been dominated by Andr\'{e} Seznec's designs, such as \cite{Seznec05}.

%\hfill mds
 
%\hfill January 11, 2007

\subsection{Problem}
\label{sec:problem}
Branch prediction has come a long way and modern branch predictors are highly accurate. For example, \cite{Seznec05} was able to achieve a mis-predict rate of under 3 mis-predicts per 1000 instruction. However, it remains an open question \textbf{whether further improvements in branch prediction accuracy are possible and, if so, what these new branch predictors might look like}.
In addressing this problem, difficulties arise from the fact that:
\subsubsection{Branch Prediction Design is Ad-hoc}
 \textbf{Branch prediction design is currently highly ad-hoc and non-systemized}. The prevalent approach is to simulate an existing design with a set of benchmarks, find the mis-predictions, then modify the design in an ad-hoc fashion to handle the mis-predictions. This is problematic, because without a systematic approach, we might only be picking at localized regions of the design space while other promising designs remain unexplored. 

\subsubsection{Current Evaluation Methodologies Ignores Hardware Realities}
There are presently \textbf{no effective way to evaluate the implementability of a predictor design}. The existing evaluation methodology simulates a predictor design in software by feeding it instruction traces from real benchmarks, then measuring the number of times the predictor would have mis-predicted, along with the number of cycles that the fetch unit would have spent on the wrong paths \cite{championship}. However, this evaluation strategy does not take into account the implementability of the design. The only implementation metric which can be extracted from a software model of the predictor is the amount of storage the design requires. Other metrics, such as power consumption, design complexity, delay, etc., are summarily ignored. Indeed, a common critism of the Championship Branch Prediction competition is that some of the submitted designs aren't even implementable in real hardware. For example, neural network based designs like \cite{Lin01} require a large adder-tree that would have unacceptably large delays in real hardware. 

\subsection{Goals}
We wish to address the problems raised in section \ref{sec:problem} by:
\begin{itemize}
  \item developing a systematic and formal methodology for designing branch predictors
  \item providing the means to evaluate the implementability of a predictor design 
\end{itemize} 

\subsection{Prior Work}
Emer and Gloy did pioneering work on this problem in \cite{Emer97}. They developed a general language in which most table-based branch predictors can be expressed. Emer's language modeled all predictors as a table with an \emph{Input} signal which indexes into the table, a \emph{Prediction} output signal, which is the value stored at the index, and a \emph{Feedback} input signal which can be fed back into the table. Figure~\ref{fig:emer} shows Emer and Gloy's model of a branch predictor.

\begin{figure}[h!]
  \centering
    \includegraphics[width=0.5\textwidth]{emer}
    \caption{Emer and Gloy's model of a branch predictor}
    \label{fig:emer}
\end{figure}

Any design expressed in Emer and Gloy's language can then be mapped to a tree structure. These tree structures now become the individuals in a genetic programming algorithm \cite{Koza92}. The genetic programming algorithm works as follows:
\begin{enumerate}
  \item an initial population is randomly generated. This forms the first generation
  \item individuals are ranked by order of their fitness. Higher ranked individuals have a higher probability of surviving to the next generation.
    \begin{itemize}
      \item fitness of an individual is evaluated by generating a software simulator of the corresponding predictor and running the simulator over instruction traces from SPEC benchmarks
    \end{itemize}
  \item genetic operations are applied to the current generation to generate the next generation. Possible genetic operations are:
    \begin{itemize}
    \item replication
    \item crossover
    \item mutation
    \item encapsulation
      \end{itemize}
    \item repeat steps 2) and 3) until the halting condition
\end{enumerate}

Additional constraints were added to ensure that we produce ``legal and usable individuals'' \cite{Emer97}. After the genetic operations step, individuals that map to illegal langauge expressions are corrected, and individuals that require too much memory have their storage structures randomly truncated until their memory requirement falls under a given limit. 

Using the methodology outlined above, Gloy and Emer were able to achieve respectable results. They were able to automatically synthesis predictors that were comparable to the state-of-the-art predictors at the time (e.g. GShare) \cite{Emer97}. 

It's also worth noting that Seznec used simulated annealing for tuning parameters in his predictor design in \cite{Seznec05}, but otherwise maintained an ad-hoc design approach.

\subsection{Contributions}
We made the following contributions in this project:
\begin{itemize}
  \item Developed a branch predictor description language that, compared to Emer and Gloy's language:
    \begin{itemize}
      \item has more high-level constructs common to all branch predictors
      \item allows for dynamic resizing of the widths of all expressions (Emer and Gloy's language forces all expression widths to be statically determined)
      \item is far more amenable to the generation of RTL 
      \item can more easily be extended to the design of other hardware structures
      \end{itemize}
  \item Developed techniques to constrain the random generation of the initial population
  \item Developed a plug-and-play infrastructure that allows for rapid evaluation of new genetic algorithms and new design languages
  %\item Emer and Gloy did their work almost 15 years ago, and had orders of magnitude less compute power than what's available today. We were able to leverage 
 % \item furthermore, there's been great advances in tools for genetic programming (i.e. \cite{ECJtools}), which can potentially be leveraged  
\end{itemize}

\section{Methodology}
In this section we explain our methodology and infrastructure.
\subsection{Language}
Our first step was to define a new description language for branch predictors. 

Our language consists of generic modules, each of which may contain any number of parameters. In our example below of a \textbf{TABLE} module, the two parameters are the number of entries in the table, and the width of each entry. For all parameters, it is possible to specify a possible range of values, using the notation [\emph{upper limit} : \emph{lower limit}]. The random generation algorithm will then pick a value from within the range. The random generation has a preference for powers of 2's. 

\begin{algorithmic}
\STATE \textbf{TABLE}\#([\emph{max \# of entries} : \emph{min \# of entries}]\%100,[\emph{max size of entry} : \emph{min size of entry}]\%100) 
\end{algorithmic}
(ignore the ``\%100'' tokens for now; we will discuss that shortly)

Next, we specify the inputs for each module. A module can have any number of inputs. Each input is either the output of another module, or one of the primitive inputs (which we will discuss shortly). Each input can be any bitstring of \emph{arbitrary} length. In fact, all variables/signals in our langauge are assumed to be bitstrings of arbitrary length.

For each input, you can specify the affinity the input has for particular types of signals. For example,  

We wanted our language to be both highly general and applicable to the design of all hardware structures, while at the same time be highly tuned specifically to the design of branch predictors. We aimed to achieve this by including both high-level, branch predictor specific language constructs, and low-level generic constructs. An example of our language is shown below:
%\begin{algorithm}
%\caption{low-level logic functions}
%\label{lst:lang}

\textbf{Basic logic:}
\begin{algorithmic}

\STATE \textbf{XOR} \{GHR\_FETCH\%20, GHR\_RETIRE\%20\};
\STATE \textbf{AND} \{null\%0, null\%0\};
\STATE \textbf{ADD} \{null\%0, null\%0\};
\STATE \textbf{OR}  \{null\%0, null\%0\};
\STATE \textbf{CONCAT} \{null\%0, null\%0\};
\STATE \textbf{MUX} \{EQUAL\%20, TABLE\_2BITCNTR\%20,
\STATE TABLE\_2BITCNTR\%20\};
\STATE \textbf{MSB} \{null\%0\};
\STATE \textbf{NOT} \{null\%0\};
\STATE \textbf{EQUAL} \{null\%0, null\%0\};
\STATE \textbf{SHIFT}\#([3:0]\%100) \{null\%0, null\%0\};
\STATE \textbf{HASH} \#([128:8]\%100)\{null\%0\};
\end{algorithmic}

\textbf{Branch prediction specific:}
\begin{algorithmic}
\STATE \textbf{GHR\_FETCH}\#([128:8]\%100);
\STATE \textbf{GHR\_RETIRE}\#([128:8]\%100);
\STATE \textbf{PHR\_FETCH}\#([128:8]\%100);
\STATE \textbf{PHR\_RETIRE}\#([128:8]\%100);
\STATE \textbf{TABLE\_2BITCNTR}\#([16384:2048]\%100) 
\STATE \{readPC\%80, writeTaken\%50, writePC\%80, writeValid\%70\};
\STATE \textbf{TABLE}\#([16384:2048]\%100,[64:0]\%100) 
\STATE \{readPC\%80, writeTaken\%50, writePC\%80, writeValid\%70\};
\STATE \textbf{TABLE\_CNTR}\#([16384:2048]\%100,[8:0]\%100) 
\STATE \{readPC\%80, writeTaken\%50, writePC\%80, writeValid\%70\};
\end{algorithmic}


%\end{algorithm}

\subsection{Fitness test}
\subsection{Genetic algorithm}

\section{Experiments and results}

\section{Analysis}

\section{Conclusion and future work}

%\subsubsection{Branch Prediction}
%Subsubsection text here.

%\subsubsection{Design Space Exploration}
%Subsubsection text here.


% An example of a floating figure using the graphicx package.
% Note that \label must occur AFTER (or within) \caption.
% For figures, \caption should occur after the \includegraphics.
% Note that IEEEtran v1.7 and later has special internal code that
% is designed to preserve the operation of \label within \caption
% even when the captionsoff option is in effect. However, because
% of issues like this, it may be the safest practice to put all your
% \label just after \caption rather than within \caption{}.
%
% Reminder: the "draftcls" or "draftclsnofoot", not "draft", class
% option should be used if it is desired that the figures are to be
% displayed while in draft mode.
%
%\begin{figure}[!t]
%\centering
%\includegraphics[width=2.5in]{myfigure}
% where an .eps filename suffix will be assumed under latex, 
% and a .pdf suffix will be assumed for pdflatex; or what has been declared
% via \DeclareGraphicsExtensions.
%\caption{Simulation Results}
%\label{fig_sim}
%\end{figure}

% Note that IEEE typically puts floats only at the top, even when this
% results in a large percentage of a column being occupied by floats.


% An example of a double column floating figure using two subfigures.
% (The subfig.sty package must be loaded for this to work.)
% The subfigure \label commands are set within each subfloat command, the
% \label for the overall figure must come after \caption.
% \hfil must be used as a separator to get equal spacing.
% The subfigure.sty package works much the same way, except \subfigure is
% used instead of \subfloat.
%
%\begin{figure*}[!t]
%\centerline{\subfloat[Case I]\includegraphics[width=2.5in]{subfigcase1}%
%\label{fig_first_case}}
%\hfil
%\subfloat[Case II]{\includegraphics[width=2.5in]{subfigcase2}%
%\label{fig_second_case}}}
%\caption{Simulation results}
%\label{fig_sim}
%\end{figure*}
%
% Note that often IEEE papers with subfigures do not employ subfigure
% captions (using the optional argument to \subfloat), but instead will
% reference/describe all of them (a), (b), etc., within the main caption.


% An example of a floating table. Note that, for IEEE style tables, the 
% \caption command should come BEFORE the table. Table text will default to
% \footnotesize as IEEE normally uses this smaller font for tables.
% The \label must come after \caption as always.
%
%\begin{table}[!t]
%% increase table row spacing, adjust to taste
%\renewcommand{\arraystretch}{1.3}
% if using array.sty, it might be a good idea to tweak the value of
% \extrarowheight as needed to properly center the text within the cells
%\caption{An Example of a Table}
%\label{table_example}
%\centering
%% Some packages, such as MDW tools, offer better commands for making tables
%% than the plain LaTeX2e tabular which is used here.
%\begin{tabular}{|c||c|}
%\hline
%One & Two\\
%\hline
%Three & Four\\
%\hline
%\end{tabular}
%\end{table}


% Note that IEEE does not put floats in the very first column - or typically
% anywhere on the first page for that matter. Also, in-text middle ("here")
% positioning is not used. Most IEEE journals/conferences use top floats
% exclusively. Note that, LaTeX2e, unlike IEEE journals/conferences, places
% footnotes above bottom floats. This can be corrected via the \fnbelowfloat
% command of the stfloats package.



%\section{Conclusion}
%The conclusion goes here.

%\section{Future Work}
%Can apply the ideas in this paper to other microarchitecture elements (data prefetcher, etc.)

% conference papers do not normally have an appendix


% use section* for acknowledgement
%\section*{Acknowledgment}

%The authors would like to thank...





% trigger a \newpage just before the given reference
% number - used to balance the columns on the last page
% adjust value as needed - may need to be readjusted if
% the document is modified later
%\IEEEtriggeratref{8}
% The "triggered" command can be changed if desired:
%\IEEEtriggercmd{\enlargethispage{-5in}}

% references section

% can use a bibliography generated by BibTeX as a .bbl file
% BibTeX documentation can be easily obtained at:
% http://www.ctan.org/tex-archive/biblio/bibtex/contrib/doc/
% The IEEEtran BibTeX style support page is at:
% http://www.michaelshell.org/tex/ieeetran/bibtex/
\bibliographystyle{IEEEtran}
% argument is your BibTeX string definitions and bibliography database(s)
\bibliography{references}
%
% <OR> manually copy in the resultant .bbl file
% set second argument of \begin to the number of references
% (used to reserve space for the reference number labels box)
%--------------------------------------------------------------------
%\begin{thebibliography}{1}
%\bibitem{IEEEhowto:kopka}
%H.~Kopka and P.~W. Daly, \emph{A Guide to \LaTeX}, 3rd~ed.\hskip 1em plus
%  0.5em minus 0.4em\relax Harlow, England: Addison-Wesley, 1999.
%----------------------------------------------------------------------
%\end{thebibliography}




% that's all folks
\end{document}


